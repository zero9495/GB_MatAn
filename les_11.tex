\documentclass{article}
\usepackage{amsfonts}
\usepackage{amsmath}
\usepackage{indentfirst}
\usepackage{mathptmx}

\title{Mathematical analysis. Lesson 11. Homework}
\author{Plyuhin Aleksandr}
\date{}

\begin{document}
\maketitle

\section{}
    \[\sum_{n=1}^{\infty} \frac{n^n}{(n!)^2} \]

    $ \lim\limits_{n \to \infty} \frac{(n+1)^{n+1}}{((n+1)!)^2} : \frac{n^n}{(n!)^2} = $
    $ \lim\limits_{n \to \infty} \frac{(n+1)^{n+1}}{((n+1)!)^2} \cdot \frac{(n!)^2}{n^n} = $
    $ \lim\limits_{n \to \infty} \frac{(n+1)^n (n+1) (n!)^2} {(n!(n+1))^2 n^n} = $
    
    $ \lim\limits_{n \to \infty} \frac{(n+1)^n} {(n+1) n^n} = $
    $ \lim\limits_{n \to \infty} \frac{1}{n+1} \cdot (\frac{n+1}{n})^n = $
    $ \lim\limits_{n \to \infty} \frac{1}{n+1} \cdot (1+\frac{1}{n})^n = $
    $ 0 < 1 $
    
    This series is convergent

\section{}
    \[\sum_{n=1}^{\infty} \frac{n}{2^n} \]
    
    $ \lim\limits_{n \to \infty} \sqrt[n]{a_n} = $
    $ \lim\limits_{n \to \infty} \sqrt[n]{\frac{n}{2^n}} = $
    $ \lim\limits_{n \to \infty} \frac{\sqrt[n]{n}}{2} = $
    $ \frac{1}{2} < 1 $
    
    This series is convergent

\section{}
    \[\sum_{n=1}^{\infty} \frac{(-1)^n}{n+\ln{n}} \]
    
    $ \lim\limits_{n \to \infty} \frac{(-1)^n}{n+\ln{n}} = 0$
    
    This series is convergent

\section{}
    \[\sum_{n=1}^{\infty} \frac{3^n}{2^n} \]
    
    $ \lim\limits_{n \to \infty} (n(\frac{3^n}{2^n}:\frac{3^{n+1}}{2^{n+1}}-1)) = $
    $ \lim\limits_{n \to \infty} (n(\frac{3^n}{2^n} \cdot \frac{2^{n+1}}{3^{n+1}}-1)) = $
    
    $ \lim\limits_{n \to \infty} (n(\frac{2}{3}-1)) = $
    $ \lim\limits_{n \to \infty} (n(-\frac{1}{3})) = -\infty < 1$
    
    This series is divergent

\section{}
    $ f(x) = \ln{(16x^2)} $
    
    $ f'(x) = \frac{1}{16x^2} 32x = \frac{2}{x} $
    
    $ f''(x) = -\frac{2}{x^2} $
    
    $ f'''(x) = \frac{4}{x^3} $
    
    $ f^{(4)}(x) = -\frac{12}{x^4} $
    
    $ f^{(n)}(x) = (-1)^{n+1} \frac{1}{x^n} 2(n-1)! $
    
    $ \frac{f^{(n)}(a)}{n!} (x-a)^n = (-1)^{n+1} \frac{1}{a^n} 2(n-1)! \frac{1}{n!} (x-a)^n = $
    
    $ = (-1)^{n+1} 2(n-1)! \frac{1}{n!} (x-1)^n = $
    
    $ = (-1)^{n+1} 2 \frac{1}{n} (x-1)^n = $
    
    $ = -1 \cdot ((-1)(x-1))^n \cdot \frac{2}{n} = $
    
    $ = -1 \cdot (1-x)^n \cdot \frac{2}{n} $

\end{document}