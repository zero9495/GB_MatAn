\documentclass{article}
\usepackage{amsfonts}

\title{Mathematical analysis. Lesson 2. Homework}
\author{Plyuhin Aleksandr}
\date{}

\begin{document}
\maketitle

\section{Introduction to Mathematical analysis}

\subsection{Execrise 1}

Elements of sequence are elements of some set.
Sequence is function of natural numbers.

\subsection{Execrise 2}

\begin{enumerate}
    \item %1
        $
        \forall y \in [0;1] : sgn(y)=1
        $
        
        For every $y$, if $y$ in [0;1], then $sgn(y)=1$.
        
        % Неверно. При нуле будет ноль.
        This statement is false, because if $y=0$, then $sgn(y)=0$.
        
        Negative: $
            \exists y \in [0;1] : sgn(y) \neq 1
        $
        
        \ 
        
        
    \item %2
        $
        \forall n \in \mathbb{N} > 2 : \exists x,y,z \in \mathbb{N} : x^n = y^n + z^n
        $
        
        For every natural number $n$, 
        there exist natural numbers x,y,z such that $x^n = y^n + z^n$.
        
        % Неверно. Это теорема Ферма.
        This statement is false. It's Fermat's Last Theorem.
        
        Negative: $
            \forall n \in \mathbb{N} > 2 : \not\exists x,y,z \in \mathbb{N} : x^n = y^n + z^n
        $
        
        \ 
        
    \item %3
        $
        \forall x \in \mathbb{R} \  \exists X \in \mathbb{R} : X>x
        $
        
        For every real number $x$, there exists real number $X$ such that $X>x$.
        
        % Верно, например X=x+1
        This statement is true. An $X$ may always be equals to $x+1$.
        
        Negative: $
            \forall x \in \mathbb{R} \  \not\exists X \in \mathbb{R} : X>x
        $
        
        \ 
        
    \item %4
        $
        \forall x \in \mathbb{C} \  \not\exists y \in \mathbb{C} : x>y || x<y
        $
        
        For every complex number $x$, there does not exist complex number $y$ such that
        $x>y || x<y$.
        
        % Верно. Комплексные числа нельзя сравнивать на больше/меньше;
        This statement is true. We can't compare two complex numbers.
        
        Negative: $
            \forall x \in \mathbb{C} \  \exists y \in \mathbb{C} : x>y || x<y
        $
        
        \ 
        
    \item %5
        $
        \forall y \in [0; \frac{\pi}{2} ] \  \exists \varepsilon >0 : \sin{y} < \sin{(y + \varepsilon)}
        $
        
        For every $y$ in $[0; \frac{\pi}{2} ]$, there exist $\varepsilon >0$ such that
        $\sin{y} < \sin{(y + \varepsilon)}$.
        
        % Неверно, т.к. в точке y=pi/2 синус принимает свое максимальное значение =1.
        This statement is false, because maximum value of sine is when $y = \frac{\pi}{2} $.
        
        Negative: $
            \exists y \in [0; \frac{\pi}{2} ] \  \not\exists \varepsilon >0 : \sin{y} < \sin{(y + \varepsilon)}
        $
        
        \ 
        
    \item %6
        $
        \forall y \in [0; \pi ) \  \exists \varepsilon >0 : \cos{y} > \cos{(y + \varepsilon)}
        $
        
        For every $y$ in $[0; \pi )$, there exist $\varepsilon >0$ such that
        $\cos{y} > \cos{(y + \varepsilon)}$.
        
        % Верно, например e=pi-y, тогда cos(y+e)=-1, а при любом y из интервала [0; pi) выполняется cos(y)>-1
        This statement is true.
        If $y$ in $[0; pi)$, then $\cos{y}$ is greater than $-1$.
        If $\varepsilon = \pi - y$, then $\cos{(y + \varepsilon)}=-1$.
        
        Negative: $
            \forall y \in [0; \pi ) \  \not\exists \varepsilon >0 : \cos{y} > \cos{(y + \varepsilon)}
        $
        
        \ 
        
    \item 
        $
        \exists x : x \not\in \{ \mathbb{N,Z,Q,R,C} \}
        $
        
        There exist $x$ that does not belong to any set of $\mathbb{N,Z,Q,R,C}$.
        
        % Верно, если мы рассматриваем множества, состоящие, например, из букв.
        This statement may be true, if we consider sets, that contains not numbers.
        
        Negative: $
            \forall x : x \in \{ \mathbb{N,Z,Q,R,C} \}
        $
        
\end{enumerate}

\end{document}