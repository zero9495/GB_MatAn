\documentclass{article}
\usepackage{amsfonts}
\usepackage{amsmath}
\usepackage{indentfirst}

\title{Mathematical analysis. Lesson 3. Homework}
\author{Plyuhin Aleksandr}
\date{}

\begin{document}
\maketitle

\section{Limit of a function}


\subsection{}
    $
    f(x)=\sin{ \frac{1}{x} } + \sin{x}
    $

\subsection{}
    $
    f(x)= 
    \begin{cases}
      \sin{ \frac{1}{x} },  & \text{for } x>0\\
      0,  & \text{for } x=0
    \end{cases}
    $

\subsection{$ f(x)=x^3-x^2 $}
    \renewcommand{\thesubsubsection}{\thesubsection.\alph{subsubsection}}
    \subsubsection{}
    $ domain=codomain=(-\infty;+\infty) $
    
    \subsubsection{}
    $ f'(x)=3x^2-2x $
    
    $ f''(x)=6x-2 $
    
    \begin{enumerate}
        \item {Root $x=1$}
            $ f(1)=0 $
            
            $ f`(1) \ne 0 $
            
            The multiplicity of a root $= 1$.
            
        \item {Root $x=0$}
        
            $ f(0)=f`(0)=0 $
            
            $ f``(0) \ne 0 $
            
            The multiplicity of a root $= 2$.
    \end{enumerate}
    
    \subsubsection{}
    $ f(x)>0,  & \text{for } x \in (1; +\infty) $
    
    $ f(x)<0,  & \text{for } x \in (-\infty; 0) \bigcup (0;1)$
    
    \subsubsection{}
    $ f`(x)=3x^2-2x=0 $
    
    $ D=b^2-4ac=4 $
    
    $ x = \frac{-b \pm \sqrt{D}}{2a} $
    
    $ x_1 = \frac{2 + 2}{6} = \frac{4}{6} $
    
    $ x_2 = \frac{2 - 2}{6} = 0 $
    
    Function is increasing on $(-\infty;0) \bigcup (\frac{4}{6};+\infty)$.
    
    Function is decreasing on $(0;\frac{4}{6})$.
    
    \subsubsection{}
    Function is neither even nor odd.
    
    \subsubsection{}
    Function isn't limited.
    
    \subsubsection{}
    Function isn't periodic.
    
\subsection{}
    \subsubsection{}
    \[
    \lim_{x \to 0} \frac{3x^3-2x^2}{4x^2} = \frac{0}{0} = 
    \lim_{x \to 0} \frac{x^2(3x-2)}{4x^2} =
    \lim_{x \to 0} \frac{3x-2}{4} = 0
    \]
    
    \subsubsection{}
    \[
    \lim_{x \to 0} \frac{\sqrt{1+x}-1}{\sqrt[3]{1+x}-1} = \frac{0}{0} = 
    \lim_{x \to 0} \frac{(1+x)^{\frac{1}{2}}-1}{(1+x)^{\frac{1}{3}}-1} = *
    \]
    
    \[
    1+x=t^6, \  \lim_{x \to 0} t = \lim_{x \to 0} \sqrt[6]{1+x} = 1
    \]
    
    \[
    * = \lim_{t \to 1} \frac{t^{\frac{6}{2}}-1}{t^{\frac{6}{3}}-1} =
    \lim_{t \to 1} \frac{t^3-1}{t^2-1} =
    \lim_{t \to 1} \frac{(t-1)(t^2+t+1)}{(t-1)(t+1)} =
    \]
    
    \[
    \lim_{t \to 1} \frac{t^2+t+1}{t+1} = \frac{3}{2}
    \]


\section{Theorems of limits}
    \renewcommand{\thesubsection}{\thesection.\alph{subsection}}
    \subsection{}
    \[
    \lim_{x \to 0} \frac{\sin{2x}}{4x} = 
    \lim_{x \to 0} \frac{1}{2} \cdot \frac{\sin{2x}}{2x} = \frac{1}{2}
    \]
    
    \subsection{}
    \[
    \lim_{x \to 0} \frac{x}{\sin{x}} = 1
    \]
    
    \subsection{}
    \[
    \lim_{x \to 0} \frac{x}{\arcsin{x}} =
    \{ \arcsin{x} \equiv  x, & \text{for } x \to 0 \} = 1
    \]
    
\end{document}